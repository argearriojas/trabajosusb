%%	Esta es la plantilla para comenzar a crear un libro de tesis.
%%	En los comentarios se pueden encontrar los diferentes comandos
%%	que se pueden usar para personalizar el trabajo.
%%
%%	La clase tesisusb está basada por la que creó Kevin Hernandez en 2005.
%%	Esta versión requiere del motor texlive-xetex, y del paquete 
%%	texlive-latex3.
%%
%%	Se dispone de un comando para insertar imágenes: \imagen[]{}
%%	se usa en forma parecida a \includegraphics, pero permite
%%	marco y sombras. El código fuente que permite hacer esto fue
%%	conseguido en http://tex.stackexchange.com/a/11483/3954

\documentclass[]{tesisusb}
%%	La clase tesisusb carga por defecto los siguientes paquetes:
%%		graphicx,siunitx,fontspec
\usepackage[]{tesisusb}
%%	El paquete tesisusb carga por defecto los siguientes paquetes:
%%		calc,setspace,fancyhdr,geometry,textcase,indentfirst,
%%		caption,subcaption,tocloft,hyperref,cite,calc,tikz.

%\usepackage[dvips]{epsfig}
%\usepackage{newlfont}
%\usepackage{enumerate}
%\usepackage{pdfpages}
%\usepackage{enumitem}
%\usepackage{url}
%\usepackage[overload]{textcase}
\usepackage{float}
\floatstyle{plaintop}
\restylefloat{table}

\hypersetup{
    pdftitle={Título del trabajo},    % title
    pdfauthor={Autor},     % author
    pdfcreator={creador del PDF},   % creator of the document
    pdfkeywords={PC1} {PC2} {PC3}, % list of keywords
    pdfnewwindow=true,      % links in new window
    colorlinks=true,       % false: boxed links; true: colored links
    linkcolor=black,          % color of internal links (change box color with linkbordercolor)
    citecolor=black,        % color of links to bibliography
    filecolor=black,      % color of file links
    urlcolor=black           % color of external links
}
			% Para incluir los paquetes extra que se necesiten
\usepackage{lipsum}

%\include{lib/funciones}		% 

%%	=================
%%	Datos del Trabajo
%%	=================

%%	Tipo de trabajo (uno de los siguientes)
%%		\tesispregrado
%%		\tesismaestria
%%		\tesisdoctoral
%%		\informepasantia

\tesispregrado

%%	Comandos disponibles
%%		\universidad{}		- Valor por defecto: Universidad Simón Bolívar
%%		\instituto{}			- En caso de que se trate de un instituto
%%		\logo{}				- Logo de la institución
%%		\decanato{}
%%		\coordinacion{}
%%		\programa{}			- Carrera. Ejm. Licenciatura en Física
%%		\titulogrado{}		- Título a obtener. Ejm. Licenciado en Física
%%		\fecha{}				- Mes y año de entrega del trabajo ante la coordinación
%%		\autor{}				- Nombre completo del estudiante
%%		\carnet{}
%%		\titulo{}			- Título del Trabajo
%%		\tutor[f ó m]{}		- Tutor del trabajo
%%		\cotutor[f ó m]{}	- Opcional
%%		\asesor[f ó m]{}		- Opcional. Pensado para asesores externos en Pasantías

\fecha{Octubre, 2013}
\logo{logoUSB.png}	% dirige al archivo logoUSB.png
\autor{Nombre1 Nombre2 Apellido1 Apellido2}
\carnet{12-34567}
\titulo{ Título del trabajo. De requerir una o más letras minúsculas, puede usar $\backslash$M\MakeLowercase{ake}L\MakeLowercase{owercase}\{\} }

\tutor[m]{Nombre Apellido}
%\tutor[f]{Gloria Buendía}
%\cotutor[m]{Manolo}
%\cotutor[f]{María}
%\asesor[m]{Ing. Pepe}
%\asesor[f]{Ing. Rosario}

%%	==================================
%%	Datos de la aprobación del trabajo
%%	==================================

%%	Generar página de aprobación
%%		\aprobacionjurado

\aprobacionjurado

%%	Comandos disponibles
%%		\fechaaprobacion{}
%%		\mencionsobresaliente{}
%%		\juradoA{texto bajo el nombre}{nombre}
%%		\juradoB{texto bajo el nombre}{nombre}
%%		\juradoC{texto bajo el nombre}{nombre}
%%		\juradoD{texto bajo el nombre}{nombre}
%%		\juradoE{texto bajo el nombre}{nombre}

\fechaaprobacion{18 de octubre de 2013}
%\mencionsobresaliente
\juradoA{Presidente}{Prof. Nombre Apellido}
\juradoB{Principal - Tutor}{Prof. Nombre Apellido}
\juradoC{Principal Externo - UCV}{Prof. Nombre Apellido}
%\juradoD{Co-tutor - USB}{Prof. Julio Walter}
%\juradoE{Asesor - Empresa tal}{Prof. José A. López}

%%	============================================
%%	Inclusión del resto de las páginas iniciales
%%	============================================

%%	Comandos disponibles
%%		\dedicatoria{}		- Dirección a archivo .tex
%%		\agradecimientos{}	- Dirección a archivo .tex
%%		\resumen{}			- Dirección a archivo .tex
%%		\palabrasclave{}		- Conjunto de palabras clave separadas por coma

\dedicatoria{lib/dedicatoria}
\agradecimientos{lib/agradecimientos}
\resumen{lib/resumen}
\palabrasclave{PC1, PC2, PC3, etc.}

\begin{document}


% ÍNDICE DE CONTENIDOS
\indicegeneral
\listadetablas
\listadefiguras

\begin{onehalfspace}
% COMIENZAN A ANEXARSE LOS CAPÍTULOS DEL LIBRO
\cuerpo

\chapter{Capítulo}

Contenido

\begin{figure}
\centering
\marco
\sombra
\imagen[width=3cm]{logoUSB}
\caption{Caption}
\end{figure}

\begin{figure}
\centering
%\marco
%\sombra
\imagen[width=3cm]{logoUSB}
\caption{Caption}
\end{figure}

\begin{figure}
\centering\marco\sombra
\imagen[width=3cm]{logoUSB}
\caption{Caption}
\end{figure}

\begin{figure}
	\begin{subfigure}[b]{0.29\textwidth}
		\centering
		\imagen[width=4cm]{logoUSB}
		\caption{Caption}
		\label{label1}
	\end{subfigure}\qquad
	\begin{subfigure}[b]{0.29\textwidth}
		\centering\marco
		\imagen[width=4cm]{logoUSB}
		\caption{Caption}
		\label{label2}
	\end{subfigure}\qquad
	\begin{subfigure}[b]{0.29\textwidth}
		\centering\sombra\marco
		\imagen[width=4cm]{logoUSB}
		\caption{Caption}
		\label{label3}
	\end{subfigure}	
	\caption{Caption}
\end{figure}

\begin{figure}
	\centering
	\begin{subfigure}[b]{0.45\textwidth}
		\centering
		\imagen[width=6cm]{logoUSB}
		\caption{Caption}
		\label{label4}
	\end{subfigure}\qquad
	\begin{subfigure}[b]{0.45\textwidth}
		\centering\marco
		\imagen[width=6cm]{logoUSB}
		\caption{Caption}
		\label{label}
	\end{subfigure}\\
	\begin{subfigure}[b]{0.45\textwidth}
		\centering\sombra
		\imagen[width=6cm]{logoUSB}
		\caption{Caption}
		\label{label5}
	\end{subfigure}\qquad
	\begin{subfigure}[b]{0.45\textwidth}
		\centering\sombra\marco
		\imagen[width=6cm]{logoUSB}
		\caption{Caption}
		\label{label6}
	\end{subfigure}	
	\caption{Caption}
\end{figure}

\section{Sección}

\subsection{Sub-sección}

\subsection{Sub-sección}

\section{Sección}

\subsection{Sub-sección}

\subsection{Sub-sección}



\chapter{Capitulo 2}
\lipsum

\apendices
%Los capítulos de aquí en adelante aparecen como apéndices

\chapter{Apéndice}
\section{sección}
\subsection{subsección}

%\bibliographystyle{unsrt-es}
%\bibliography{referencias.bib}

\end{onehalfspace}
\end{document}

%$a \ b \; c \: d \, e \! f$
